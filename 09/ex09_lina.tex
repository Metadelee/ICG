\documentclass{article}
\usepackage{amsmath}
\usepackage{amssymb}
\usepackage[svgnames]{xcolor}
\usepackage{graphicx}
\usepackage{enumitem}
\usepackage{multicol}
\usepackage{bbm}

\title{Computer Graphics: Assignment 09} % Title

\author{Lina Gundelwein, Letitia Parcalabescu, Anushalakshmi Manila} % Author name

\date{\today} % Date for the report

\begin{document}

\maketitle 

\section{Perception}
<<<<<<< HEAD
\begin{enumerate}
\item Red and green are opponent colors such that they cannot be combined with each other: There is one set cells reactive to red light and one set reactive to blue light. The red-reactive neurons fire when red light enters the retina, indicating we look at something red. If they do not fire, it indicates that we look at something green. As they cannot fire and not fire at the same time there is no reddish green. A reddish yellow however is possible as yellow is perceived by another set of cells.
\item Very dim stars have a intensity to low to be perceived by cones, only by the more light sensitive rods. As there are no rods in the center of the eye, the astronomer cannot so the star when focusing on it, but rather have to look out of the corner of there eye.
\item Metamers are different light spectra that are perceived as identical for a given observer. Metamers occur because each cone type responds to a broad range of wavelengths, and there can be lights with different wavelength combinations that produce an equivalent receptor response and thereby color perception.
\end{enumerate}

\section{3D Visual Perception}
\begin{itemize}
\item Anaglyph images are an overlay of two images from slightly different perspectives with different color filters, typically one red and one cyan. To see the 3D image one has to look through glasses which have a cyan color filter on one eye and a red color filter on the other side. Each eye will just see one of the images (as you cannot see the cyan image through the cyan glass) and therewith only one of the perspectives. The brain fuses the two perspective images to a 3D image
\item Polarized light vibrates on only one plane. Similar to the two images that make up anaglyph images, a movie can be recorded with two cameras from slightly different perspectives and can then be projected through two differently polarized filters. The viewer has to wear glasses with two accordingly polarized lenses, such that one eye sees one perspective and the other one the slightly different one.
\end{itemize}

\section{Color Systems}
\begin{enumerate}
\item 
\begin{itemize}
\item additive:  combination of colors by addition of the spectra, e.g. CRT monitors, multiple projectors on one screen
\item subtractive: combination of colors by multiplication of the spectra, e.g. photographic film, color pens, color printer
\end{itemize}
\item CMYK: (0,1,1,0) HSV: (0, 1, 1) HSL: (0,1,0.5) (formulas see slides)
\item to produce a pure black: mixture of all colors mostly looks brownish and printing only one black ink is less expensive than printing 3 different color layers
\end{enumerate}
=======
\begin{itemize}
\item Due to the reason that opposite opponent colors are never perceived together, we can percieve reddish yellow, but not reddish green. 
\item Astronomers can not focus on dim stars because...

\item Metamers are a pair of colors with differing spectral compositions but generate the same or similar tristimulus color values under at least one set of viewing conditions such as lighting, size and angle of viewing or the chromatic sensitivity of observers. \\
Different reasons for metamerism include  trichromatic nature of human vision, the sensitivity of the observer and the lighting level.
\end{itemize}

\section{3D visual perception}
\begin{itemize}
	\item Stereoscopic display: The basic technique of is to present offset images that are displayed separately to the left and the right eye. Both of these 2D offset images are then combined in the brain to give the perception of 3D depth.
	
	\item Anaglyph 3D: Anaglyph 3D images contain two differently filtered colored images, one for each eye. When viewed through the "color-coded" "anaglyph glasses", each of the two images reaches the eye it is intended for, revealing an integrated stereoscopic image. The visual cortex of the brain fuses this into perception of a three-dimensional scene or composition.
	
	\item Polarized light: To create stereoscopic vision, 3D films are captured using two lenses placed side by side, just like our eyes. 3D glasses with blue and red filters ensured viewers’ left and right eyes saw the correct image. Modern 3D films use polarised light instead of red and blue light. Modern 3D films use polarised light instead of red and blue light.  Images destined for viewers' left eyes are polarised on a horizontal plane, whereas images destined for their right eyes are polarised on a vertical plane.
\end{itemize}
\section{Color systems}
\begin{itemize}
	\item Combination of colors by addition of spectra is known as additive color mixing and that of by multiplication of spectra is called subtractive color mixing. Computer monitors and televisions are an application of additive color. Photographic film, color pens, color printer are subtractive color applications.
	\item Representation of the RGB color:
	\begin{itemize}
		\item CMYK: C = 1 - R; M = 1 - G; Y = 1 - B
		\item HSV: H $\in [ 0^0, 360^0]$; S = (max - min) / max; V = max(r,g,b)
		\item HSL: H $\in [ 0^0, 360^0]$; S = (max - min) / max; L = 0.5*max(r,g,b) + 0.5*min(r,g,b)
	\end{itemize}
	\item Most of the current printers still have a cartridge for black besides the color cartridges because...
\end{itemize}
>>>>>>> 73ee62b02e4d23807dea91dbcc0dd5c40ea421f0
\end{document}