\documentclass{article}
\usepackage{amsmath}
\usepackage{amssymb}
\usepackage[svgnames]{xcolor}
\usepackage{graphicx}
\usepackage{enumitem}
\usepackage{multicol}
\usepackage{bbm}

\title{Computer Graphics: Assignment 09} % Title

\author{Lina Gundelwein, Letitia Parcalabescu, Anushalakshmi Manila} % Author name

\date{\today} % Date for the report

\begin{document}

\maketitle 

\section{Perception}
\begin{itemize}
\item Due to the reason that opposite opponent colors are never perceived together, we can percieve reddish yellow, but not reddish green. 
\item Astronomers can not focus on dim stars because...

\item Metamers are a pair of colors with differing spectral compositions but generate the same or similar tristimulus color values under at least one set of viewing conditions such as lighting, size and angle of viewing or the chromatic sensitivity of observers. \\
Different reasons for metamerism include  trichromatic nature of human vision, the sensitivity of the observer and the lighting level.
\end{itemize}

\section{3D visual perception}
\begin{itemize}
	\item Stereoscopic display: The basic technique of is to present offset images that are displayed separately to the left and the right eye. Both of these 2D offset images are then combined in the brain to give the perception of 3D depth.
	
	\item Anaglyph 3D: Anaglyph 3D images contain two differently filtered colored images, one for each eye. When viewed through the "color-coded" "anaglyph glasses", each of the two images reaches the eye it is intended for, revealing an integrated stereoscopic image. The visual cortex of the brain fuses this into perception of a three-dimensional scene or composition.
	
	\item Polarized light: To create stereoscopic vision, 3D films are captured using two lenses placed side by side, just like our eyes. 3D glasses with blue and red filters ensured viewers’ left and right eyes saw the correct image. Modern 3D films use polarised light instead of red and blue light. Modern 3D films use polarised light instead of red and blue light.  Images destined for viewers' left eyes are polarised on a horizontal plane, whereas images destined for their right eyes are polarised on a vertical plane.
\end{itemize}
\section{Color systems}
\begin{itemize}
	\item Combination of colors by addition of spectra is known as additive color mixing and that of by multiplication of spectra is called subtractive color mixing. Computer monitors and televisions are an application of additive color. Photographic film, color pens, color printer are subtractive color applications.
	\item Representation of the RGB color:
	\begin{itemize}
		\item CMYK: C = 1 - R; M = 1 - G; Y = 1 - B
		\item HSV: H $\in [ 0^0, 360^0]$; S = (max - min) / max; V = max(r,g,b)
		\item HSL: H $\in [ 0^0, 360^0]$; S = (max - min) / max; L = 0.5*max(r,g,b) + 0.5*min(r,g,b)
	\end{itemize}
	\item Most of the current printers still have a cartridge for black besides the color cartridges because...
\end{itemize}
\end{document}