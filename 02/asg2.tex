\documentclass{article}
\usepackage{amsmath}
\usepackage{amssymb}
\usepackage[svgnames]{xcolor}
\usepackage{graphicx}
\usepackage{enumitem}
\usepackage{multicol}
\usepackage{bbm}

\title{Computer Graphics: Assignment 02} % Title

\author{Lina Gundelwein, Letitia Parcalabescu, Anushalakshmi Manila} % Author name

\date{\today} % Date for the report

\begin{document}

\maketitle 

\section{Preprocessor} 
With the definition \verb|#define SQUARE(a) a*a| g and h return wrong results, because the macro only does the textual replacement and thus you end up with:
\begin{align*}
&g(1) = 1-1*1-1 = -1\\
&g(2) = 1-2*1-2 = -3\\
&...\\
&h(1) = 1./1*1 = 1\\
&h(2) = 1./2*2 = 1\\
&...
\end{align*}

\section{Pointers, Arrays, and All the Rest}
The matrix parameter requires a double pointer because it is a pointer to an array and an array is referenced by a pointer. $\rightarrow$ thus a double pointer is needed
%TODO I think this is wrong


\section{Output Devices}
Generating the whole array may take $T = \frac{1}{50Hz} = 0.02\,s$. Generating one pixel may thus take $ T_p = \frac{1}{1000 \cdot 1000 \cdot 50Hz} = 2\times 10^{-8}\,s$.

\section{Lighting Models}
\begin{itemize}
\item ambient: The general brightness of the scene, intensity value will be added to each pixel
\\diffuse: Light reflection such that an incident ray is reflected at many angles following the lambertian emission law. The surface will have the same radiance from all angles.
\\specular:  Light reflection such that the angle of incidence roughly equals the angle of reflection
\item The exponent describes the decay of reflectivity when differing from the main reflection direction $R=2N(NL)-L$. A high exponent leads to a mirror-like reflection while a low exponent leads to a rough looking surface.
\item To get an intensity gradient, one interpolates between the face normals. To preserve sharp edges, multiple face normals are stored per vertex.
\end{itemize}
\end{document}